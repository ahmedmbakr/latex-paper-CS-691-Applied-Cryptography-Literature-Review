\section{Introduction}
\label{sec:introduction}

% What is blockchain in data marketplace
Blockchain technology has revolutionized the way data is managed and exchanged, particularly in data marketplaces.
By enabling secure and tamper-proof transactions between data buyers and sellers, blockchain technology provides a transparent and trustworthy platform for data exchange.
This allows for greater data privacy and ownership, as well as more efficient and cost-effective data transactions.
Additionally, blockchain technology can also enable new business models and opportunities for data monetization, allowing individuals and organizations to profit from their data in a secure and controlled manner.
As a result, blockchain technology in data marketplaces has the potential to significantly transform various industries, from healthcare to \ac{iot}, by enabling greater data collaboration and unlocking new insights and value.

% The use of blockchain and marketplace in medical and IoT applications
Blockchain technology has the potential to revolutionize data management in medical and \ac{iot} applications.
In the medical domain, a pharmaceutical corporation spends around 12 million dollars every year to acquire anonymized medical data from several sources.
As the need for clinical research continues to grow, it is becoming increasingly apparent that innovative medical data marketplaces are required to ensure the safe and efficient exchange of medical data and records between medical data sellers and interested buyers.
Comparatively, in \ac{iot} applications, the rising cost of data transmission makes it impractical for data owners to handle data trading on a local level.
Therefore, it is a promising solution for data owners to store
and trade the \ac{iot} data at a powerful data center.

% Marketplace Requirements
Marketing requirements govern the interactions between sellers and buyers through the marketplace.
\ac{gdpr} requires data marketing to be transparent and reliable and its requirements are three-fold.
First, the data seller her the right to be informed when a data buyer wants to buy his data, the means of using  her data, or the completion of the trading transaction.
Second, the data seller has the right to control access to her data by granting access to accept/deny requests from data buyers.
Third, the identity of the data sellers shall be preserved.

% Problems that face blockchain-based fair data trading
While blockchain technology offers many benefits in the data marketplace, there are also several challenges that must be addressed. 
One major challenge is scalability, as blockchain technology can be slow and resource-intensive, which can limit its ability to handle large volumes of data.
Another challenge is ensuring data accuracy, as the source and validity of data must be verified to prevent fraudulent or misleading data.
Additionally, compliance with \ac{gdpr} requirements is a challenging hurdle.
Moreover, verifying the availability of data to the buyer without revealing the plain data is a significant challenge in order to ensure fairness for the buyer to pay only when he receives the correct data and to secure the seller's payment by the end of the exchange transaction.
Finally, regulatory challenges also exist, as the use of blockchain technology in the data marketplace is still relatively new, and regulatory frameworks are still evolving, making it difficult to navigate legal and compliance issues.

% What is presented in this paper
As indicated above, despite the tremendous benefits brought by implementing blockchain in the data marketplace, the infrastructure is still confronted with many security and privacy challenges.
In this paper, we present a survey of current research that addresses the aforementioned trading issues using blockchain in medical and \ac{iot} applications.
In \cref{sec:blockchain-marketplace-trading-schemes}, we show how some of the challenges are solved by three different schemes.
In each scheme, we present the problem definition and network model, the contributions, the threat model, the used primitives, and an overview of the solution.
While in section \cref{sec:schemes-comparison}, we present a summary table-based comparison between the described schemes.
Finally, in \cref{sec:conclusions}, we conclude the work and depict where the research will go in the future in this area.
