\section{Schemes Comparison}
\label{sec:schemes-comparison}

In this section, we compare the three presented papers.
It is also worth mentioning that all three papers use smart contract-based blockchain networks.
In~\cref{tab:general-comparison}, we present a general comparison between 
the three papers in terms of GDPR requirements compliance, domain, and blockchain characteristics.
Moreover, \cref{tab:used-primitives-comparison} shows the primitives used in each paper.

\begin{table*}[h!]\centering
\caption{General schemes comparison}\label{tab:general-comparison}
\scriptsize
\begin{threeparttable}
\begin{tabular}{lrrrrrrrr}\toprule
&R2I\tnote{1} &R2C\tnote{2} &Hide Seller's Identity &Fine Grained Data &Domain &Blockchain Type &On/Off-chain Model \\\midrule
\cite{xue2023blockchain} &Yes &Yes &Yes &Yes &Medical &Not mentioned\tnote{4} &On-chain \\
\cite{liu2022blockchain} &Yes &Yes &Yes (conditional)\tnote{3} &No &\ac{iot} &Consortium &On/Off-chain \\
\cite{alsharif2020blockchain} &Yes &Yes &No &No &Medical &Public &On/Off-chain \\
\bottomrule
\end{tabular}
\begin{tablenotes}
    \item[1] R2I: Right of the Seller to be informed
    \item[2] R2C: Right of the seller to control her listed data
    \item[3] Conditional means: with the ability to expose the misbehaving seller
    \item[4] We see that the blockchain can be public or private
\end{tablenotes}
\end{threeparttable}
\end{table*}

\begin{table*}[h!]\centering
\caption{Used primitives comparison}\label{tab:used-primitives-comparison}
\scriptsize
\begin{threeparttable}
\begin{tabular}{lrrrrrrrr}
\toprule % Add a line on top of the table
&\ac{zkp} &Signature &PVSS &Symmetric Encryption &Asymmetric Encryption &CP-ABE &Merkle Hash Tree \\\midrule
\cite{xue2023blockchain} &Fiat-Shamir &Structure preserving signature &No &\ac{aes} &Elgamal &No &Yes \\
\cite{liu2022blockchain} &Fiat-Shamir &PS Signature &Yes &\ac{aes} &Elgamal &No &No \\
\cite{alsharif2020blockchain} &ZK-SNARK &Yes\tnote{1} &No &\ac{aes} &No &Yes &No \\
\bottomrule
\end{tabular}
\begin{tablenotes}
    \item[1] No specific signature scheme was mentioned
\end{tablenotes}
\end{threeparttable}
\end{table*}

